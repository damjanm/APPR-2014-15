\documentclass[11pt,a4paper]{article}

\usepackage[slovene]{babel}
\usepackage[utf8x]{inputenc}
\usepackage{graphicx}
\usepackage{pdfpages}
\usepackage{hyperref}

\pagestyle{plain}

\begin{document}
\title{Poročilo pri predmetu \\
Analiza podatkov s programom R}
\author{Damjan Manevski}
\maketitle

\section{Izbira teme}

  Tema mojega proekta je Sistemi za souporabo koles. V njem bom obravnaval podatke, pridobljene iz ameriškega podjetja Capital Bikeshare, ki velja za eden od največjih sistemov na svetovni ravni in se ukvarja z izposojo koles v kratkem časovnem obdobju (podobno sistemu Bicikelj v Ljubljani). V nadaljevanju bomo pogledali koliko se uporabljajo te sisteme v različnih državah. 
  
  Cilj projekta je raziskati možnosti optimizacije za izposojo koles na podlagi številk polnih/praznih postajališč, številk uporabnikov v določenem obdobju in številk poškodovanih koles.
  
  Povezave do podatkov:
  
\begin{enumerate}

\item{\url{http://cabidashboard.ddot.dc.gov/cabidashboard/#Home}}

\item{\url{https://archive.ics.uci.edu/ml/datasets/Bike+Sharing+Dataset}}

\item{\url{http://www.capitalbikeshare.com/system-data}}

\end{enumerate}

\section{Obdelava, uvoz in čiščenje podatkov}

Uporabil sem 4 tabele. Tri tabele sem uvozil v CSV obliki, četrta pa iz spletne strani kot html. Preden sem jih uvozil, sem pobrisal nepotrebne vrstice in stolpce.

Na spodnjih grafih sem prikazal in primerjal uporabo koles med tednom. Lahko sklepamo, da ni bistvene razlike v uporabi, oziroma da število uporabljenih koles se ne spreminja v različnih dnevih. Drugi graf ponazarja, da se večinoma uporabnikov nahajajo v Washington D.C, vendar pa je podjetje prisotno tudi v nekaterih drugih ameriških mestah. Na tretjem grafu pa vidimo, da se je trend uporabe takšnih sistemov povečal v zadnjih letih.


\includepdf[pages={1-3}]{../slike/grafi1.pdf}


\section{Analiza in vizualizacija podatkov}

V tej fazi sem naredil dva zemljevida. Potrebno je bilo dodati nekaj podatkov, ki se razlikujejo od začetnih. Na prvem zemljevidu sem zanemaril Kitajsko (ki je sicer nesporni lider v sistemih te vrste), da bi dobil boljši zemljevid. Na drugem zemljevidu sem razdelil države v 3 skupinah (v odvisnosti od števila mest, ki uporabljajo tega sistema) in vidimo, da to ustreza rezultatu, ki je prikazan na prvemu zemljevidu.

\includegraphics[width=\textwidth]{../slike/zemljevid1.pdf}

\includegraphics[width=\textwidth]{../slike/zemljevid2.pdf}


% 
% \section{Napredna analiza podatkov}
% 
% \includegraphics{../slike/naselja.pdf}

\end{document}
