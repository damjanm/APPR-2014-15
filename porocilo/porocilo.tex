\documentclass[11pt,a4paper]{article}

\usepackage[slovene]{babel}
\usepackage[utf8x]{inputenc}
 \usepackage[unicode]{hyperref}
\usepackage{graphics}
\usepackage{pdfpages}
\usepackage{hyperref}
\usepackage{animate}
\usepackage[font=small,skip=0pt]{caption}
\usepackage{float}


\pagestyle{plain}

\begin{document}
\title{Poročilo pri predmetu \\
Analiza podatkov s programom R \\
Sistemi za souporabo koles}
\author{Damjan Manevski}
\maketitle

\section{Izbira teme}


  Tema mojega proekta je Sistemi za souporabo koles. V njem bom obravnaval podatke, pridobljene iz ameriškega podjetja Capital Bikeshare, ki velja za eden od največjih sistemov na svetovni ravni in se ukvarja z izposojo koles v kratkem časovnem obdobju (podobno sistemu Bicikelj v Ljubljani). V nadaljevanju bomo pogledali koliko se uporabljajo te sisteme v različnih državah. 
  
  Cilj projekta je raziskati možnosti optimizacije za izposojo koles na podlagi številk  uporabnikov v določenem obdobju ter pogledati stanje na svetovni ravni.
  
  Povezave do podatkov:
  
\begin{enumerate}

\item{\url{http://cabidashboard.ddot.dc.gov/cabidashboard/#Home}}

\item{\url{https://archive.ics.uci.edu/ml/datasets/Bike+Sharing+Dataset}}

\item{\url{http://www.capitalbikeshare.com/system-data}}

\end{enumerate}

\section{Obdelava, uvoz in čiščenje podatkov}

Uporabil sem 5 tabel. Štiri tabele sem uvozil v CSV obliki, peta pa iz spletne strani kot html. Preden sem jih uvozil, sem pobrisal nepotrebne vrstice in stolpce.

\begin{enumerate}

\item{tabela: Prikazuje podatke o opazovanjih v letih 2011 in 2012. Vsebuje naslednje spremenljivke: date (imenska), season (številska, letni čas; 1-spomlad, 2-poletje, 3-jesen, 4-zima), year (številska, 0 - 2011, 1 - 2012), month (številska, od 1 do 12), holiday (številska, 0 - ni praznik, 1 - je praznik ), weekday (številska, od 0 do 6), workingday (številska, 1 - delovni dan, 0 - sicer ), weathersit (številska, od 1 do 4, vremensko stanje), temp (številska, temperatura v Celzij med 0 in 1), atemp (številska, občutna temperatura med 0 in 1), hum (številska, vlaga, med 0 in 1),	windspeed (številska, moč vetra, med 0 in 1),	casual (številska, število občasnih uporabnikov),	registered (številska, število registriranih uporabnikov),	cnt (številska, casual+registered).}

\item{tabela: Prikazuje podatke o opazovanjih v letu 2014. Vsebuje naslednje spremenljivke: Date (imenska), Municipality	(imenska), Trip.Duration (imenska; razdeli potovanja v nekoliko skupinah, odvisno od tega koliko so trajala),	Membership.Type (imenska),	No..of.Trips (številska),	Miles.Travelled (številska).}

\item{tabela: Prikazuje podatke o opazovanjih v juliju 2014. Vsebuje naslednje spremenljivke: Year (imenska), Month (imenska), Station.Name (imenska),	Municipality (imenska), Departures (številska, število odhodov),	Arrivals (številska, število prihodov).}

\item{tabela: Prikazuje stanje na svetovni ravni. Vsebuje naslednje spremenljivke:  City (imenska),	Country (imenska),	Continent (imenska), Name (imenska, ime podjetja),	System (imenska, ime sistema),	Status (imenska, Active/Closed),	Year.inaugurated (številska, leta otvoritev),	Stations (številska, število postajališč), Bicycles (številska, število koles).}

\item{tabela: Podobna prvi tabeli, le da vsebuje še številsko spremenljivko hour. Ker vsak dan (vsako vrstico) iz prve tabele sem razdelil na 24 vrstic, tabela je nekaj večja, vendar pa nisem imel težav z njo delati.}

\end{enumerate}

Na spodnjih grafih sem prikazal in primerjal uporabo koles med tednom. Lahko sklepamo, da ni bistvene razlike v uporabi, oziroma da se število uporabljenih koles ne spreminja v različnih dnevih. Uporaba v različnih urah enem dnevu se pa razlikuje. Tretji graf ponazarja, da se večinoma uporabnikov nahajajo v Washington D.C, vendar pa je podjetje prisotno tudi v nekaterih drugih ameriških mestah. Na četrtem grafu pa vidimo, da se je trend uporabe takšnih sistemov povečal v zadnjih letih.

\makebox[\textwidth][l]{
\includegraphics[width=0.9\textwidth]{../slike/grafi1.pdf}
}
Iz iste tabele sem pogledal dejavniki, ki vplivajo na uporabo koles in sem posledično izračunal kovarianco med naslednjimi spremenljivkami:
\begin{itemize}
\item{Kovarianca med temperaturo in število uporabnikov: 0.6310657;}
\item{Kovarianca med moč vetra in število uporabnikov: -0.234545}
\item{Kovarianca med vlaga in število uporabnikov: -0.1006586}
\end{itemize}
Dobili smo pričakovane rezultate. 

\includepdf{../slike/grafi3.pdf}

\includepdf[pages={1-2}]{../slike/grafi1a.pdf}


\section{Analiza in vizualizacija podatkov}

V tej fazi sem naredil dva zemljevida. Potrebno je bilo dodati nekaj podatkov, ki se razlikujejo od začetnih. Na prvem zemljevidu sem razdelil države v 3 skupinah (v odvisnosti od števila mest, ki uporabljajo tega sistema). Na drugem zemljevidu sem zanemaril Kitajsko (ki je sicer nesporni lider v sistemih te vrste), da bi dobil boljši zemljevid. Vidimo, da to ustreza rezultatu, ki je prikazan na prvemu zemljevidu.

Čeprav iz zemljevidov ni razvidno, s pomočjo statisničnih orodjih sem ugotovil, da so podatke razpršene. Po izračunu sem dobil, da je mediana 790, sredina 3593.204, ter standardni odklon 7690.308. Torej lahko sklepamo, da večinoma držav imajo majhno število razpoložljivih koles. 

\makebox[\textwidth][l]{
\includegraphics[width=0.9\textwidth]{../slike/zemljevid2.pdf}
}

\newpage
\includegraphics[width=\textwidth]{../slike/zemljevid1.pdf}

\section{Napredna analiza podatkov}

  V zadnji fazi sem pogledal kakšne zaključke lahko izvlečemo iz tabel, ki sem jih obravnaval. Uporabil sem metode, ki smo jih imeli na vajah (prileganje), nasvete iz zadnje predstavitve, ter orodja, ki so bili dani v virih na spletni učilnici.

  Najprej sem narisal zemljevid, ki prikazuje razmerje med številom razpoložljivih koles in številom prebivalcev, torej zemljevid podoben drugemu zemljevidu iz prejšnje faze. Dobili smo:
\newpage
\begin{figure}[t]
\makebox[\textwidth][c]{
\includegraphics[width=1.1\textwidth]{../slike/zemljevid3.pdf}
}
\end{figure}
\makebox[\textwidth][c]{
\includegraphics[width=1\textwidth]{../slike/grafi4.pdf}
}

\newpage


Ker iz zemljevida ni razvidno katere države imajo najboljše razmere, sem naredil graf na kateremu se vidijo prvih 5 držav. 

Poleg tega, sem uporabil še metode za prileganje podatkom o novih odprtih sistemih skozi čas. Dobil sem naslednji graf:
\begin{figure}[h]
\makebox[\textwidth][c]{
\includegraphics[width=1.4\textwidth]{../slike/grafi2.pdf}
}
\end{figure}

Ker ne moremo grobo oceniti katera metoda je najboljša izračunal sem vsote residualov in sem dobil:

\begin{itemize}
\item{Linearna metoda: 2259.0128}
\item{Kvadratna metoda: 1055.2984}
\item{Funkcija loess: 802.6469}
\item{Funkcija gam: 791.2780}
\end{itemize}

Sklepamo, da je funkcija \verb|gam| najbolj ustrezna za prileganje podatkom.


\begin{center}
\animategraphics[controls,loop,width=1.2\linewidth]{1}{../slike/anim1}{}{}\\
\end{center}

Za konec sem naredil animacijo, ki prikazuje kje so se odpirali nove sisteme v odbodju 1995-2014. Pri tem sem razvrstil podatke po letih in sem število novih koles v vsaki državi normiral, s ciljem da bo velikost točk odvisna od števila koles.


\newpage
\section{Zaključek}

Pogledali smo si stanje in trendi na svetovni ravni o sistemih za souporabo koles, navade, ki jih imajo kolesarji, ter dejavniki, ki vplivajo na uporabo. Pri tem sem poskusil uporabiti čim več različnih načinov analize podatkov.


\newpage
\begin{thebibliography}{3}
\bibitem{1}
  \url{http://cabidashboard.ddot.dc.gov/cabidashboard/#Home}\\
\bibitem{2}
  \url{https://archive.ics.uci.edu/ml/datasets/Bike+Sharing+Dataset}\\
\bibitem{3}
  \url{http://www.capitalbikeshare.com/system-data}\\

\end{thebibliography}

\end{document}
